\chapter{Referencial Teórico}

O presente capítulo apresenta a base teórica fundamental para a estruturação e o desenvolvimento do presente trabalho de pesquisa. Nesse contexto, essa base teórica gira em torno de conceitos relacionados aos sistemas de georreferenciamento de tempo real, \textit{internet of thing}, bancos de dados relacionais e bancos de dados não relacionais.

\section{Sistemas de georreferenciamento de tempo real}

Os sistemas de georreferenciamento ou sistemas de mapas virtuais podem conter duas bases a serem utilizadas: as dinâmicas e as estáticas\cite{MacEachren}. Nas dinâmicas, os pontos de visualização variam de acordo com novas inserções de dados que podem ser cartográficos, envolvendo latitudes e longitudes; Já nas bases de georreferenciamento estáticas, os dados geralmente são representados por imagens ou gráficos armazenados cuja variação representa situações distintas\cite{MacEachren}, como se fossem realizadas várias fotografias em diferentes locais.

Existem, também, outros tipos de dados coletados por sistemas de georreferenciamento, como é o caso....
\section {\textit{Internet of things}}
De acordo com pesquisadores, pode-se definir Internet Das Coisas como um ambiente de objetos físicos interconectados com a internet através de sensores pequenos e embutidos, criando um ecossistema de computação onipresente (ubíqua), voltado para a facilitação do cotidiano das pessoas\cite{Magrani-2018}.

Embora se tenha um entendimento sobre o que é Internet das Coisas, do inglês \textit{Internet Of Things}, com abreviação IoT, existem divergências no que se refere ao conceito. Isso porque para alguns pesquisadores o termo é associado às redes de computadores e aos seus protocolos de comunicação. Para outros a presença da inteligência artificial é fundamental nos dispositivos conectados e, ainda, os que defendem a inter-relação entre as máquinas e pessoas como fundamento do conjunto IoT\cite{Magrani-2018}.

 
\section{Bancos de dados relacionais}
Para entendermos como serão utilizados os registros das informações neste trabalho, será exposto, neste subitem noções de como as informações computacionais são armazenadas e processadas. Para isto devemos apresentar conceitos sobre Dados, Banco de Dados e Sistema de Gerenciamento de Banco de Dados.

Os termos "dado" e "informação" são semelhantes. Certos autores os diferenciam, se referindo a Dado como valores fisicamente registrados no banco de dados, e Informação como sendo o "significado" desses valores para algum usuário \cite{C.Date}.

O significado de "Dado" também pode ser associado a registro, pois é assim que os sistemas de gerenciamento de Banco de Dados tratam as informações de entrada\cite{Kotaro-2005}. Mais adiante iremos abordar com mais detalhes os Sistemas de Gerenciamento de Banco de Dados.

Os Dados precisam ser organizados em um conjunto de informações para serem armazenados e, posteriormente, podem servir para diversos fins como consultas, estatísticas, cálculos, dentre outros. O local em que eles ficam armazenados é conhecido como Banco de Dados\cite{Kotaro-2005}.
\section {NoSQL}

\section{Mineração de Dados e \textit{Big data}}

\chapter{Internet das Coisas - Internet Of Things}

Para entendermos como a Internet Das Coisas foi utilizada nesta pesquisa, primeiramente faremos um apanhado histórico acerca dos principais acontecimentos que resultaram no que conhecemos como as "coisas conectadas" ou dispositivos conectados e, veremos, como essa tecnologia se torna uma tendência para um novo modelo tanto de negócios quanto do cotidiano.

O termo Internet Das Coisas ou "Internet Of Things", em inglês, foi popularmente apresentado, de acordo com relatos por um cientista da Inglaterra chamado Kevin Ashton que, durante uma conferência da Procter \& Grambler, em 1999, falou sobre a troca de dados, em tempo real, por dispositivos, usando a Internet. Mais tarde, em 2009, Ashton escreveu um artigo, intitulado: "A Coisa Internet Das Coisas", em que descreveu como foi sua ideia para descrever o termo que ficou mundialmente conhecido e que é referência para estudos e negócios.

Em seu texto Ashton conta que trabalhava em uma empresa de cosméticos na parte de estoque e ficava chateado de como o setor era impreciso quando se precisavam fazer consultas dos produtos. Ele logo constatou que seria preciso manter um funcionário que ficasse exclusivamente anotando as informações de entrada e saída dos produtos das prateleiras para que tivesse um resultado satisfatório. Porém, isso seria muito custoso para a empresa gerando uma grande perda de tempo e dinheiro só para obter essa precisão de informações. 

Quando Kevin Ashton teve conhecimento sobre as redes sem fio, as RFIDs, teve a ideia de que as "coisas" poderiam se conectar, não mais somente os computadores. E essas coisas ou objetos poderiam transmitir as informações em tempo real, substituindo o trabalho que era de uma pessoa, por um processo automático do objeto, de forma precisa e economizando tempo.


  

\chapter{A Importância das Tecnologias da Computação na Segurança Pública}

Este capítulo abordará questões de como a informática e tecnologias afins têm marcado a dinâmica de trabalho das Áreas de Seguranças desde a inserção de ferramentas digitais até sensores e robôs para a obtenção de melhores resultados.

Para que  possamos discorrer sobre a importância de várias ferramentas computacionais na Segurança Pública, de forma específica, precisamos fazer uma breve contextualização, tanto na atuação das forças de Segurança Pública no Brasil quanto contextualizar também quais novas tecnologias vêm sendo utilizadas como forma de contenção da criminalidade em diversos países através da História. 

\section{Atuação da Segurança Pública no Brasil após o Regime Militar}

Depois que o último presidente militar João Figueiredo, que governou o país, em 1985, o Brasil passou por uma mudança estrutural, sendo promulgada uma nova Constituição, em 1988 e, com ela, novas regras e novos acontecimentos surgiram na sociedade.
A Segurança Pública, de acordo com um artigo publicado por Moema Dutra Freire, em 2012, existem paradigmas na Área Segurança Brasileira que, mesmo com a transição da Ditadura Militar para a Democracia, perduram com iniciativas que chegam a se mostrar ineficientes e ultrapassadas. São três paradigmas:

"Segurança Nacional, vigente durante a Ditadura Militar, Segurança Pública, que se fortalece com a promulgação da Constituição de 1988 e; Segurança Cidadã, perspectiva que tem se ampliado em toda a América Latina e começa a influenciar o debate em segurança no Brasil, a partir de meados de 2000" (FREIRE, 2012).

A autora ainda afirma que esses três paradigmas coexistem, ou seja, o surgimento de um não excluiu o outro e a existência deles perpassa governos, constituições e, até mesmo, ideologias fazendo com que haja uma grande influência no comportamento dos agentes de segurança que, por sua vez, exercem um papel fundamental na "sensação de segurança" divulgada por meios de comunicação, como é o caso de jornais   

\section{Contextualização das ferramentas computacionais na sociedade brasileira}
A presença das redes sociais no cotidiano da população atual é indispensável quando pensamos em fontes de informação, tanto para saber novidades sobre qual time ganhou ontem, quanto para saber o que o presidente dos Estados Unidos afirmou sobre um determinado assunto. Entretanto as redes sociais foram ganhando sua popularidade aos poucos. De acordo com Eric Lassar 1994, as redes sociais apareceram com aplicativos simples de conversação entre grupos fechados ainda por militares norte-americanos, durante o período da Segunda Guerra Mundial, em que o principal objetivo era manter uma comunicação estratégica que não fosse interceptada pelos inimigos. Mais tarde, esses aplicativos chamados de "chat" no inglês, cuja a tradução seria conversa, foram usados por estudantes universitários até que, em 2000, um cientista chamado Mário Porto, escreveu uma obra sobre o fenômeno dos aplicativos de conversação por computadores, o que não somente era troca de mensagens, mas uma rede social.

Com o advento do Facebook, em 2002 e do Twitter, em 2003, as redes sociais ganharam um grande espaço nas rotinas sociais, passando por uma explosão de visualizações que mudaram a forma de como as pessoas buscavam informações e notícias, pois eram muito mais rápidas e selecionadas do que outros meios de comunicação como TV, rádio, jornais impressos e, até mesmo, sites oficiais de notícias, o que desenvolveu problemas de confiabilidade e crimes digitais, que veremos mais adiante. Ainda abordando esse assunto, podemos citar a compra do aplicativo Whatsapp pelo grupo Facebook, em 2006, fato que fez o aplicativo de celular se tornar a principal ferramenta de troca de informações e dados da segunda década do século XXI, de acordo com dados de uma pesquisa apontada por uma revista científica, em 2008, a ComputerScience, que abordou tanto o número de aparelhos celulares, quando os dispositivos SMART conectados e destacou os principais softwares usados nesses dispositivos, dando destaque para o Facebook e o Whatsapp.

Dessa forma podemos inferir que a rede de computadores e o uso de aplicativos de celulares estão cada vez mais presentes no cotidiano e nas áreas que envolvem ciência/pesquisa e negócios, sendo bastante explorada para que possamos analisar com maiores seus efeitos e, mais a frente, abordarmos as principais formas de se utilizar o conjunto entre essas tecnologias como ferramentas do experimento científico que contemplamos neste trabalho.

   

