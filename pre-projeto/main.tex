% Modelo de trabalho acadêmico (Teses, Dissertações, TCC)
% Documento principal
%
% Universidade Federal do Maranhão - UFMA
% Autor: Sidney Cerqueira <cerqueirasidney@gmail.com>
% Modelo modificado do Cefet-MG
%
% Informações:
%   Codificação utilizada: UTF-8
%   Tamanho da tabulação: 4 (espaços)
\documentclass[oneside]{abntex2-cefetmg}            % Imprimir apenas frente
%\documentclass[doubleside]{abntex2-cefetmg}        % Imprimir frente e verso

% Importações de pacotes
%\usepackage[alf, abnt-emphasize=bf, bibjustif, recuo=0cm, abnt-etal-cite=2, abnt-etal-list=0]{abntex2cite}  % Citações padrão ABNT
\bibliographystyle{ieeetr}
%\usepackage{lmodern}
\usepackage[utf8]{inputenc}                         % Acentuação direta
\usepackage[T1]{fontenc}                            % Codificação da fonte em 8 bits
\usepackage{graphicx}                               % Inserir figuras
\usepackage{amsfonts, amssymb, amsmath}             % Fonte e símbolos matemáticos
\usepackage{booktabs}                               % Comandos para tabelas
\usepackage{verbatim}                               % Texto é interpretado como escrito no documento
\usepackage{multirow, array}                        % Múltiplas linhas e colunas em tabelas
\usepackage{indentfirst}                            % Indenta o primeiro parágrafo de cada seção.
\usepackage{microtype}                              % Para melhorias de justificação?
\usepackage{palatino}                               % Usa a fonte Palatino
\usepackage[algoruled, portuguese]{algorithm2e}     % Escrever algoritmos
\usepackage{float}                                  % Utilizado para criação de floats
%\usepackage[bottom]{footmisc}                      % Mantém as notas de rodapé sempre na mesma posição
%\usepackage{times}                                 % Usa a fonte Times
%\usepackage{lmodern}                               % Usa a fonte Latin Modern
\usepackage{subfig}                                % Posicionamento de figuras
%\usepackage{scalefnt}                              % Permite redimensionar tamanho da fonte
%usepackage{color, colortbl}                       % Comandos de cores
%\usepackage{lscape}                                % Permite páginas em modo "paisagem"
%\usepackage{ae, aecompl}                           % Fontes de alta qualidade
%\usepackage{picinpar}                              % Dispor imagens em parágrafos
\usepackage{latexsym}                              % Símbolos matemáticos
%\usepackage{upgreek}                               % Fonte letras gregas
\usepackage[table,xcdraw]{xcolor}
%\usepackage{subfigure}                              % Pacote para colocar figuras lado a lado
% Definição de comando para colocar dados alinhados a esquerda em tabelas largas com tamanhos variados
\newcolumntype{C}[1]{>{\let\newline\\\arraybackslash\hspace{0pt}}m{#1}}
% Definição de comando para centralizar dados em tabelas largas
\newcolumntype{K}[1]{>{\centering\let\newline\\\arraybackslash\hspace{0pt}}m{#1}}
\newcommand{\tabitem}{~~\llap{\textbullet}~~}    % comando para colocar bullet em item dentro de uma tabela

\newcolumntype{L}{>{\centering\arraybackslash}m{3cm}} % Definindo o tamanho de coluna em tabela

% Inclui o preâmbulo do documento
%
% Documento: Preâmbulo
%

\titulo{Análise do algoritmo FakeCheck em notícias regionais do Amapá}
%\title{Title in English}
%\subtitulo{Subtítulo do trabalho}
\autor{Rodrigo Santos Balieiro}
\local{Macapá}
\data{Outubro de 2019}
\instituicao{Universidade Federal do Amapá
\\ Bacharelado em Ciência da Computação}
\programa{Departamento de Ciências Exatas e Tecnológicas}
\tipotrabalho{Trabalho de Conclusão de Curso}
\preambulo{Anteprojeto de TCC apresentado à Universidade Federal do Amapá como requisito parcial para obtenção do título de Bacharel em Ciência da Computação.}
\orientador{Jose Walter Cárdenas Sotil}
%\orientador[Orientadora:]{Nome da orientadora}
%\coorientador{Dr. Denivaldo Lopes Pavão}
%\coorientador[Coorientadora:]{Nome da coorientadora}
%\linhapesquisa{Sistemas Inteligentes}


% Define as cores dos links e informações do PDF
\makeatletter
\hypersetup{
    portuguese,
    colorlinks,
    linkcolor=black,
    citecolor=black,
    filecolor=blue,
    urlcolor=black,
    breaklinks=true,
    pdftitle={\@title},
    pdfauthor={\@author},
    pdfsubject={\imprimirpreambulo},
    pdfkeywords={abnt, latex, abntex, abntex2}
}
\makeatother

% Redefinição de labels
\renewcommand{\algorithmautorefname}{Algoritmo}
\def\equationautorefname~#1\null{Equa\c c\~ao~(#1)\null}

% Cria o índice remissivo
\makeindex

% Início do documento
\begin{document}

    % Retira espaço extra obsoleto entre as frases.
    \frenchspacing

    % Elementos pré textuais
    \pretextual
    \include{./01-elementos-pre-textuais/capa}              % Capa
    \include{./01-elementos-pre-textuais/folhaRosto}        % Folha de rosto
    %
% Documento: Folha de aprovação
%

\makeatletter
\begin{folhadeaprovacao}

    \begin{center}
        {\large\normalfont\scshape\textbf\imprimirautor}
    \end{center}

    \vspace*{20pt}

    \begin{center}
        \ABNTEXchapterfont\Large\scshape\imprimirtitulo
        \abntex@ifnotempty{\imprimirsubtitulo}{%
            {\ABNTEXchapterfont\Large\scshape: }{\ABNTEXchapterfont\large\scshape\imprimirsubtitulo}
        }
    \end{center}

    \vspace*{15pt}

    \abntex@ifnotempty{\imprimirpreambulo}{%
        \SingleSpacing
        \begin{tabular}{p{.24\textwidth}p{.15\textwidth}p{.44\textwidth}}
            & \multicolumn{2}{p{.65\textwidth}}{\small\hyphenpenalty=10000{\imprimirpreambulo}} \\ & & \\
        \end{tabular}
    }

    \vspace*{70pt}

    \begin{center}
        Trabalho aprovado. \imprimirlocal, $\qquad$ de $\qquad \qquad$ de 2019.
    \end{center}
	
	\vspace*{20pt}
    \begin{center}
        \assinatura{\textbf{\imprimirorientador} \\ Orientador}
        \assinatura{\textbf{Msc. Marco Antônio Leal da Silva} \\ Universidade Federal do Amapá}
        \assinatura{\textbf{Esp. Adeildo Telles da Silva} \\ Universidade Federal do Amapá}
        %\assinatura{\textbf{Professor} \\ Convidado 3}
        %\assinatura{\textbf{Professor} \\ Convidado 4}
    \end{center}

    \vspace*{\fill}

%    \begin{center}
 %       \normalfont\scshape{\imprimirinstituicao}\\
  %      \normalfont\scshape{\imprimirprograma}\\
   %     \normalfont\scshape{\imprimirlocal}\\
   %     \normalfont\scshape{\imprimirdata}
   % \end{center}
 \begin{center}
        \normalfont\scshape{\imprimirlocal}\\
        \normalfont\scshape{\imprimirdata}
    \end{center}

\end{folhadeaprovacao}
\makeatother
    % Folha de aprovação
	%%
% Documento: Dedicatória
%

\begin{dedicatoria}

Dedico este trabalho a comunidade acadêmica de Ciência da Computação, a qual deixo esta singela contribuição bibliográfica.

Em especial a minha esposa Aline e meu filho Paulo Henrique pelo apoio moral e forte inspiração. 

\end{dedicatoria}
       % Dedicatória
	%%
% Documento: Agradecimentos
%

\begin{agradecimentos}

Inicialmente, agradeço à Deus, pela vida, saúde e livramentos que até aqui me mantiveram para o desenvolvimento deste trabalho...

A Família pelo apoio, carinho e sacrifício nas diversas noites em que me ausentei para as pesquisas...

Ao Orientador por abraçar a causa e dedicar-se junto a mim nos mais diversos desafios que foram proporcionados pelo tema...

Aos Amigos que sempre estiveram presentes nas adversidades e alegrias...

À Universidade Federal do Amapá -- UNIFAP, ...

Finalmente, agradeço à todos que contribuíram de forma direta ou indireta para a construção desse trabalho. 

\end{agradecimentos}
    % Agradecimentos
    %%
% Documento: Epígrafe
%

\begin{epigrafe}

\textit{“O perigo de verdade não é que computadores passem a pensar como humanos, mas sim que humanos passem a pensar como computadores”. -- Sydney Harris}

\end{epigrafe}
          % Epígrafe
    %
% Documento: Resumo (Português)
%

\begin{resumo}
	
Com a popularização do acesso às redes sociais e o consumo de notícias em tempo real, as \textit{Fake News} surgem como um grande fator que influencia cada vez mais a opinião pública e as decisões corporativas. Frente à grande disseminação em massa de conteúdo falso, se tem a tentativa de detectar a veracidade dessas informações com ferramentas virtuais como o \textit{FakeCheck} -- da USP. Assim sendo, questiona-se a efetividade de ferramentas como a \textit{FakeCheck}, analisando-a e submetendo-a a testes específicos.
	
\textbf{Palavras-chave}: Fake news, notícias falsas, complexidade de algoritmos.

\end{resumo}
          % Resumo na língua vernácula
    %
% Documento: Resumo (Inglês)
%

\begin{resumo}[Abstract]



\textbf{Keywords}: Fake News. Algorithm Complexity. Virtual Bots.

\end{resumo}
          % Resumo em língua estrangeira
    %\include{./01-elementos-pre-textuais/listaFiguras}      % Lista de figuras
    %\include{./01-elementos-pre-textuais/listaTabelas}      % Lista de tabelas
    %\include{./01-elementos-pre-textuais/listaQuadros}      % Lista de quadros
    %\include{./01-elementos-pre-textuais/listaAlgoritmos}   % Lista de algoritmos
    %%
% Documento: Lista de abreviaturas e siglas
%

\begin{siglas}
    \item[IBM] International Business Machines - Empresa Multinacional de Informática.
    \item[IOT] Internet Of Things - Internet das Coisas.
    \item[MQTT] Message Queuing Telemetry Transport - Protocolo de Telemetria e transporte de mensagens.
    \item[API] Application Programming Interface - Interface de Programação de Aplicativos.
    \item[Javascript] Linguagem de Programação fracamente tipada voltada para Sistemas Web.
    \item[WEB] Sigla em inglês para referir-se a rede mundial de computadores.
    \item[SQL] Service Query Langage - Linguagem Específica de Programação para Banco de Dados
    
\end{siglas}
       % thiago - Lista de abreviaturas e siglas
    %\include{./01-elementos-pre-textuais/listaSimbolos}     % Lista de símbolos
    \include{./01-elementos-pre-textuais/sumario}           % Sumário

    % Elementos textuais
    \textual
   	%
% Documento: Introdução
%

\chapter{Introdução} 

Nos últimos anos do séc. XXI, com o advento do uso expressivo de redes sociais\cite{IBGE-internet}, tanto por empresas, órgãos públicos ou mesmo por populares, as notícias e informativos estão presentes continuamente em diversos aparelhos, tornando a comunicação intensiva e dinâmica.

Entretanto, em meio a inúmeros textos, imagens e vídeos que "bombardeiam" o cotidiano das pessoas existem interesses comerciais, ideológicos, religiosos, dentre outros\cite{UFSC-fake-news-perigos} que, passam conteúdos manipulados ou falsos para que sejam consolidados, em que a verdadeira intenção seja o compartilhamento dessas informações para que haja uma grande repercussão.

Uma expressão em inglês que tornou-se bastante debatida nos últimos anos é a "Fake News", que em uma tradução livre, quer dizer Notícias Falsas. Ela ganhou notoriedade após as eleições presidenciais de 2016, nos Estados Unidos da América\cite{Fake-eleicoes-usa}, em que o candidato Donald Trump foi eleito em uma campanha tomada por acusações e ofensas entre os presidenciáveis, mas não pessoalmente e sim através das redes sociais. Ambos evidenciaram o fato de que "bombardeamento" das \textit{Fake News} teriam sido feitas por robôs digitais, os chamados \textit{Bots}. No Brasil, na última eleição presidencial, em 2018, teve-se uma experiência semelhante.

Sobre os robôs digitais, que são desenvolvidos com inteligência artificial e automação, pode-se questionar a complexidade de desenvolvimento de robôs digitais, bem como seus efeitos, sendo que o desafio estaria em identificá-los para conter a propagação de notícias falsas e auxiliar na tomada de decisões e opinião das pessoas que têm acesso às informações digitais.

Em meio a ferramentas criadas para detecção de notícias e conteúdos falsos, o pesquisador Rafael Augusto Monteiro, da Universidade de São Paulo, desenvolveu um protótipo capaz de tomar como entrada um texto e devolver uma resposta se o mesmo é verdadeiro ou falso. Diante disso, o presente trabalho busca analisar a efetividade da pesquisa de Monteiro em um contexto específico de notícias regionais, analisando seu algoritmo.

\chapter {Contexto}

O filósofo Platão\cite{Platao} falava em suas obras que a dúvida era essencial para chegarmos até a verdade. Seguindo o raciocínio, em meio a inúmeras fontes de informação presentes no cotidiano das pessoas é importante reconhecer o que de fato é real ou apenas especulação ou, até mesmo, a mentira, o que pode gerar a desinformação e influenciar na tomada de decisões.

As redes sociais estão cada vez mais presentes entre as fontes de conteúdo provenientes da internet\cite{Livro-Fake-News}. De acordo a pesquisa científica do Instituto Brasileiro de Geografia e Estatística (IBGE)\cite{IBGE-internet}, entre os anos de 2016 e 2017 o percentual de utilização da internet subiu de 69,3\% para 74,9\%, significando 3 em cada 4 domicílios brasileiros que possuem aparelhos conectados à internet, como tablets, celulares, smart TVs. Os usuários desses aparelhos consultam as redes sociais todos os dias para consumir informações de entretenimento e editoriais jornalísticos\cite{Livro-Fake-News}.

\textit{Fake News} é um termo usado para definir notícias
de conteúdo falso, enganoso, fabricado ou manipulado, cujas
fontes ou a declaração citada do conteúdo da notícia não pode
ser constatado como verdadeiro\cite{Uso-abuso-midia-fake-news}.

Uma vez conectadas à internet, as pessoas são expostas à uma série de fontes de informações que podem conter diversos assuntos generalizados ou especializados com um determinado fim, em que existe o agente produtor (origem), os usuários que compartilham (proliferação) e a maneira como são publicadas (tom)\cite{origem-proliferacao-fake}. Geralmente as corporações governamentais, jornalísticas, publicitárias e comerciais se utilizam das redes sociais para apresentar seus conteúdos, entretanto, existe uma parte dessas fontes de informação produzida por pessoas mal intencionadas que podem praticar crimes virtuais, propagar ideologias, distorcer informações e confundir a realidade para quem recebe a informação\cite{Economia-do-fake-news}.

Os pesquisadores Nir Kshetri e Jeffrey Voas\cite{Economia-do-fake-news} afirmam que as notícias falsas produzidas e mantidas de forma profissional têm consequências na economia, política e na sociedade, pois o consumo delas pode gerar um círculo vicioso em que o excesso da mentira torna-se verdade.

No ano de 2019 um ex-candidato à presidência da república no Brasil foi condenado por ter, supostamente, pago à empresas de mídia especializada, para impulsionarem notícias falsas que pudessem prejudicar a campanha de seu adversário, o que pode ser analisado como o poder público tem dado importância nos efeitos que o conteúdo digital pode causar quando manipulado profissionalmente\cite{Folha-multa-haddad,Globo-multa-haddad,Veja-multa-haddad}.

A detecção de \textit{fake news} passa por técnicas computacionais que compreendem redes neurais, classificação de linguagem natural, engenharia reversa, autômatos, IA e diferentes tipos de busca. Os \textit{Bots} da internet são programas que executam tarefas de forma automática. Inicialmente foram criados para facilitar respostas comuns a muitos questionamentos que as empresas recebiam, mas passaram a ser utilizados para criar perfis falsos e disseminar a desinformação nas redes sociais\cite{bots-na-internet}.

Com várias fontes de informação propagadas na internet existem cientistas e  acadêmicos que desenvolvem protótipos de programas para a detecção de notícias falsas, utilizando diversas metodologias para obter um melhor resultado. Questiona-se a eficácia e a confiabilidade que os programas de detecção de "Fake News" conseguem.


 
\chapter{Problemática}

As corporações donas das redes sociais têm desenvolvido ferramentas e estratégias para detecções de notícias falsas, principalmente com colaboração da comunidade participante da própria rede social. Entretanto, conforme o teor e a forma em que tais notícias são repassadas, questiona-se a efetividade da detecção e neutralização das "Fake News" através dos algoritmos desenvolvidos, como é o caso da ferramenta desenvolvida pela USP, chamada de "FakeCheck".

\chapter{Objetivos}

\section{Objetivo Geral}
Verificar a eficácia do algoritmo de detecção de notícias falsas \textit{Fake Check}, da USP.

\section {Objetivos Específicos}

O presente trabalho de pesquisa tem os seguintes objetivos específicos:

\begin{itemize}
	\item Analisar a complexidade do algoritmo usada na ferramenta de detecção de Fake News "FakeCheck" nas modalidades propostas pelos desenvolvedores.
	\item Testar o algoritmo do software "FakeCheck" na detecção de notícias falsas com um banco de dados proposto.
	\item Comparar resultados de busca à "fake news" pelo programa "FakeCheck" e criar resultados estatísticos.
	\item Identificar os métodos de detecção de notícias falsas e compará-los pelos usados no programa "FakeCheck".
	 
\end{itemize}

\chapter {Hipótese}

Os algoritmos vêm apresentando uma alta complexidade tanto na criação de robôs virtuais automáticos para propagar notícias falsas quanto na busca e detecção dos mesmos, tornando um processo de conflito pela identificação e neutralização de ações mal-intencionadas.

\chapter {Justificativa}

A busca por atividades virtuais maliciosas é uma prática antiga, desde a existência dos ataques virtuais, seja por meio de vírus, \textit{worms}, \textit{trojans}, \textit{adwares}, dentre outras ferramentas criadas para prejudicar os usuários de aparelhos eletrônicos. Os antivírus foram criados para tal tarefa.

Com o passar dos anos e aumento significativo no acesso à internet, os golpes e os softwares maliciosos foram aperfeiçoados com novos recursos apresentados pela área de Tecnologia da Informação, passando a conter automação, inteligência artificial e computação em nuvem a fim de atingirem seus objetivos.

Com tantos recursos disponíveis para prejudicar o usuários das tecnologias da informação, tem-se um desafio da comunidade científica em identificar, prevenir e banir todo conteúdo malicioso com ferramentas de contra-ataque e estratégias de divulgação preventivas. Todavia necessita-se estudar e analisar essas ferramentas para avaliar sua eficácia e um melhor acesso a todo conteúdo consumido na internet, sobretudo nas redes sociais.

Se os resultados de uma pesquisa acerca da eficácia na busca por notícias falsas apresentarem resultados tanto positivos ou negativos, esta pesquisa poderá apresentar uma contribuição para as pesquisas tecnológicas.


\chapter {Metodologia}

Como o que se pretende fazer seja estudar tanto os softwares que impulsionam as notícias falsas quanto os que combatem-nas, uma metodologia será entrevistar alguns responsáveis pelo projeto que desenvolve sistemas web e de aplicativos.

O tipo de entrevista que se pretende é do tipo subjetiva, em que os questionamentos serão acerca do desenvolvimento dos softwares, bem como sua eficiência.

Uma comparação pode ser feita entre as ferramentas de detecção de \textit{fake news} para analisar os resultados.


\chapter{Cronograma}

O presente trabalho é pautado na realização das atividades listadas, como segue:

\begin{enumerate}
	\item Revisão bibliográfica;
	\item Investigação dos objetivos;
	\item Identificação de recursos técnicos;
	\item Entrega do anteprojeto; 
	\item Análise do algoritmo.
	\item Teste do algoritmo com banco de dados próprio.
	\item Apresentação final do trabalho de pesquisa para a conclusão do curso.
	
\end{enumerate}

As atividades acima são efetuadas conforme o cronograma descrito na Tabela \ref*{tab-cronograma}. 

\begin{table}[h!]\begin{center}
		\caption{Cronograma}\label{tab-cronograma}
		\begin{tabular*}{\textwidth}{@{\extracolsep{\fill}} c c c c c c c c c c c c}
			\toprule
			& Etapa & nov. & dez. & jan & fev. & mar\\
			\midrule
			& 1 & x & x & x & x & x &\\
			& 2 & x & x &  &  &  &\\
			& 3 & x & x & x & x &  &\\
			& 4 &  &  &  & x & x &\\
			& 5 &  &  &  &  & x &\\
			& 6 &  &  & x & x & x &\\
			& 7 &  &  &  &  &  &  &\\
			
			\bottomrule                             
		\end{tabular*}
\end{center}\end{table}
   	%\chapter{Referencial Teórico}

O presente capítulo apresenta a base teórica fundamental para a estruturação e o desenvolvimento do presente trabalho de pesquisa. Nesse contexto, essa base teórica gira em torno de conceitos relacionados aos sistemas de georreferenciamento de tempo real, \textit{internet of thing}, bancos de dados relacionais e bancos de dados não relacionais.

\section{Sistemas de georreferenciamento de tempo real}

Os sistemas de georreferenciamento ou sistemas de mapas virtuais podem conter duas bases a serem utilizadas: as dinâmicas e as estáticas\cite{MacEachren}. Nas dinâmicas, os pontos de visualização variam de acordo com novas inserções de dados que podem ser cartográficos, envolvendo latitudes e longitudes; Já nas bases de georreferenciamento estáticas, os dados geralmente são representados por imagens ou gráficos armazenados cuja variação representa situações distintas\cite{MacEachren}, como se fossem realizadas várias fotografias em diferentes locais.

Existem, também, outros tipos de dados coletados por sistemas de georreferenciamento, como é o caso....
\section {\textit{Internet of things}}
De acordo com pesquisadores, pode-se definir Internet Das Coisas como um ambiente de objetos físicos interconectados com a internet através de sensores pequenos e embutidos, criando um ecossistema de computação onipresente (ubíqua), voltado para a facilitação do cotidiano das pessoas\cite{Magrani-2018}.

Embora se tenha um entendimento sobre o que é Internet das Coisas, do inglês \textit{Internet Of Things}, com abreviação IoT, existem divergências no que se refere ao conceito. Isso porque para alguns pesquisadores o termo é associado às redes de computadores e aos seus protocolos de comunicação. Para outros a presença da inteligência artificial é fundamental nos dispositivos conectados e, ainda, os que defendem a inter-relação entre as máquinas e pessoas como fundamento do conjunto IoT\cite{Magrani-2018}.

 
\section{Bancos de dados relacionais}
Para entendermos como serão utilizados os registros das informações neste trabalho, será exposto, neste subitem noções de como as informações computacionais são armazenadas e processadas. Para isto devemos apresentar conceitos sobre Dados, Banco de Dados e Sistema de Gerenciamento de Banco de Dados.

Os termos "dado" e "informação" são semelhantes. Certos autores os diferenciam, se referindo a Dado como valores fisicamente registrados no banco de dados, e Informação como sendo o "significado" desses valores para algum usuário \cite{C.Date}.

O significado de "Dado" também pode ser associado a registro, pois é assim que os sistemas de gerenciamento de Banco de Dados tratam as informações de entrada\cite{Kotaro-2005}. Mais adiante iremos abordar com mais detalhes os Sistemas de Gerenciamento de Banco de Dados.

Os Dados precisam ser organizados em um conjunto de informações para serem armazenados e, posteriormente, podem servir para diversos fins como consultas, estatísticas, cálculos, dentre outros. O local em que eles ficam armazenados é conhecido como Banco de Dados\cite{Kotaro-2005}.
\section {NoSQL}

\section{Mineração de Dados e \textit{Big data}}

\chapter{Internet das Coisas - Internet Of Things}

Para entendermos como a Internet Das Coisas foi utilizada nesta pesquisa, primeiramente faremos um apanhado histórico acerca dos principais acontecimentos que resultaram no que conhecemos como as "coisas conectadas" ou dispositivos conectados e, veremos, como essa tecnologia se torna uma tendência para um novo modelo tanto de negócios quanto do cotidiano.

O termo Internet Das Coisas ou "Internet Of Things", em inglês, foi popularmente apresentado, de acordo com relatos por um cientista da Inglaterra chamado Kevin Ashton que, durante uma conferência da Procter \& Grambler, em 1999, falou sobre a troca de dados, em tempo real, por dispositivos, usando a Internet. Mais tarde, em 2009, Ashton escreveu um artigo, intitulado: "A Coisa Internet Das Coisas", em que descreveu como foi sua ideia para descrever o termo que ficou mundialmente conhecido e que é referência para estudos e negócios.

Em seu texto Ashton conta que trabalhava em uma empresa de cosméticos na parte de estoque e ficava chateado de como o setor era impreciso quando se precisavam fazer consultas dos produtos. Ele logo constatou que seria preciso manter um funcionário que ficasse exclusivamente anotando as informações de entrada e saída dos produtos das prateleiras para que tivesse um resultado satisfatório. Porém, isso seria muito custoso para a empresa gerando uma grande perda de tempo e dinheiro só para obter essa precisão de informações. 

Quando Kevin Ashton teve conhecimento sobre as redes sem fio, as RFIDs, teve a ideia de que as "coisas" poderiam se conectar, não mais somente os computadores. E essas coisas ou objetos poderiam transmitir as informações em tempo real, substituindo o trabalho que era de uma pessoa, por um processo automático do objeto, de forma precisa e economizando tempo.


  

\chapter{A Importância das Tecnologias da Computação na Segurança Pública}

Este capítulo abordará questões de como a informática e tecnologias afins têm marcado a dinâmica de trabalho das Áreas de Seguranças desde a inserção de ferramentas digitais até sensores e robôs para a obtenção de melhores resultados.

Para que  possamos discorrer sobre a importância de várias ferramentas computacionais na Segurança Pública, de forma específica, precisamos fazer uma breve contextualização, tanto na atuação das forças de Segurança Pública no Brasil quanto contextualizar também quais novas tecnologias vêm sendo utilizadas como forma de contenção da criminalidade em diversos países através da História. 

\section{Atuação da Segurança Pública no Brasil após o Regime Militar}

Depois que o último presidente militar João Figueiredo, que governou o país, em 1985, o Brasil passou por uma mudança estrutural, sendo promulgada uma nova Constituição, em 1988 e, com ela, novas regras e novos acontecimentos surgiram na sociedade.
A Segurança Pública, de acordo com um artigo publicado por Moema Dutra Freire, em 2012, existem paradigmas na Área Segurança Brasileira que, mesmo com a transição da Ditadura Militar para a Democracia, perduram com iniciativas que chegam a se mostrar ineficientes e ultrapassadas. São três paradigmas:

"Segurança Nacional, vigente durante a Ditadura Militar, Segurança Pública, que se fortalece com a promulgação da Constituição de 1988 e; Segurança Cidadã, perspectiva que tem se ampliado em toda a América Latina e começa a influenciar o debate em segurança no Brasil, a partir de meados de 2000" (FREIRE, 2012).

A autora ainda afirma que esses três paradigmas coexistem, ou seja, o surgimento de um não excluiu o outro e a existência deles perpassa governos, constituições e, até mesmo, ideologias fazendo com que haja uma grande influência no comportamento dos agentes de segurança que, por sua vez, exercem um papel fundamental na "sensação de segurança" divulgada por meios de comunicação, como é o caso de jornais   

\section{Contextualização das ferramentas computacionais na sociedade brasileira}
A presença das redes sociais no cotidiano da população atual é indispensável quando pensamos em fontes de informação, tanto para saber novidades sobre qual time ganhou ontem, quanto para saber o que o presidente dos Estados Unidos afirmou sobre um determinado assunto. Entretanto as redes sociais foram ganhando sua popularidade aos poucos. De acordo com Eric Lassar 1994, as redes sociais apareceram com aplicativos simples de conversação entre grupos fechados ainda por militares norte-americanos, durante o período da Segunda Guerra Mundial, em que o principal objetivo era manter uma comunicação estratégica que não fosse interceptada pelos inimigos. Mais tarde, esses aplicativos chamados de "chat" no inglês, cuja a tradução seria conversa, foram usados por estudantes universitários até que, em 2000, um cientista chamado Mário Porto, escreveu uma obra sobre o fenômeno dos aplicativos de conversação por computadores, o que não somente era troca de mensagens, mas uma rede social.

Com o advento do Facebook, em 2002 e do Twitter, em 2003, as redes sociais ganharam um grande espaço nas rotinas sociais, passando por uma explosão de visualizações que mudaram a forma de como as pessoas buscavam informações e notícias, pois eram muito mais rápidas e selecionadas do que outros meios de comunicação como TV, rádio, jornais impressos e, até mesmo, sites oficiais de notícias, o que desenvolveu problemas de confiabilidade e crimes digitais, que veremos mais adiante. Ainda abordando esse assunto, podemos citar a compra do aplicativo Whatsapp pelo grupo Facebook, em 2006, fato que fez o aplicativo de celular se tornar a principal ferramenta de troca de informações e dados da segunda década do século XXI, de acordo com dados de uma pesquisa apontada por uma revista científica, em 2008, a ComputerScience, que abordou tanto o número de aparelhos celulares, quando os dispositivos SMART conectados e destacou os principais softwares usados nesses dispositivos, dando destaque para o Facebook e o Whatsapp.

Dessa forma podemos inferir que a rede de computadores e o uso de aplicativos de celulares estão cada vez mais presentes no cotidiano e nas áreas que envolvem ciência/pesquisa e negócios, sendo bastante explorada para que possamos analisar com maiores seus efeitos e, mais a frente, abordarmos as principais formas de se utilizar o conjunto entre essas tecnologias como ferramentas do experimento científico que contemplamos neste trabalho.

   


    \include{./03-elementos-pos-textuais/referencias}
    

\end{document}
